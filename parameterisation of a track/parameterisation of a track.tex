\documentclass{article}
\usepackage{amsmath, amssymb}
\usepackage{graphicx}

\begin{document}

\section*{Parameterisation of a Race Track}
According to Mathias, an easy way to describe the motion of a car on a race track is to use the so called curvilinear coordinate system. The first step to use curvilinear coordinate system is to find a parameterisation of the centre line of the track by arc length of it.

Here, $\gamma_{c}(s)$ maps the arc length $s$ to the Cartesian coordinate of the track
\[
\gamma_{\text{c}}(s) = 
\begin{pmatrix}
x_{\text{r}}(s) \\
y_{\text{r}}(s)
\end{pmatrix}
\].

However, it is almost impossible to find an elegant mathematical representation of the centre line, and also, real-world censor often provides discrete data, for example, point clouds. Therefore, one way to solve the problem is to use a set of ordered points to describe the centre line and interpolate between the points.
\subsubsection*{The formulation of this problem is:}

\begin{itemize}
    \item \textbf{Given:} A set of ordered points $(x_i, y_i)$, $i = 1, \dots, n$
    \item \textbf{Output:} A smooth curve that goes through every point. In particular, the curve needs to be:
    \begin{itemize}
        \item $C^2$ smooth at each $(x_i, y_i)$
    \end{itemize}
\end{itemize}

\begin{center}
\includegraphics[width=0.5\textwidth]{IMG_272C181FA208-1.jpeg} % Optional sketch if you want to include one
\end{center}

\subsubsection*{Solution: Cubic Spline}

A 3rd-order polynomial, which has 4 free parameters, can be used to interpolate between two points such that It satisfies 4 equations for each piece:
\[
\begin{cases}
\gamma_{i}(s_i) = \gamma_{i+1}(s_{i}) = (x_{i+1}, y_{i+1}) \\
\gamma_{i}(s_{i-1}) = \gamma_{i-1}(s_{i-1}) = (x_{i}, y_{i}) \\
\gamma_{i}'(s_i) = \gamma_{i+1}'(s_i) \\
\gamma_{i}''(s_i) = \gamma_{i+1}''(s_i)
\end{cases}
\]

Then
\[
\gamma(s)=
\begin{cases}
\gamma_1(s), & s_0 \leq s \leq s_1 \\
\gamma_2(s), & s_1 \leq s \leq s_2 \\
\vdots
\end{cases}
\]
\bigskip

is $C^2$ smooth.

\end{document}